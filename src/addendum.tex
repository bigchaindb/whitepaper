\documentclass[a4paper]{article}
\usepackage[utf8]{inputenc}
\usepackage{lipsum}% for pagenumbering
\usepackage{hyperref}

\title{Addendum to the BigchainDB Whitepaper}
\author{ascribe GmbH, Berlin, Germany}

\begin{document}

\pagenumbering{gobble}% Remove page numbers (and reset to 1)

\maketitle

This addendum summarizes significant changes since the BigchainDB white\-paper was last updated. The online \href{https://bigchaindb.readthedocs.io/}{BigchainDB Documentation} is kept up-to-date.

\begin{itemize}
  \item There are more details about how BigchainDB achieves \href{https://bigchaindb.readthedocs.io/en/latest/topic-guides/decentralized.html}{decentralization} and \href{https://bigchaindb.readthedocs.io/en/latest/topic-guides/immutable.html}{immutability / tamper-resistance} in the BigchainDB Documentation.
  \item \textbf{Section 4, Section 6 and Figure 11.} It's possible that not every node will vote on every block. Only a quorum of votes is required. See \href{https://github.com/bigchaindb/bigchaindb/issues/190}{Issue \#190 on GitHub}.
  \item \textbf{Section 4.3, Section 6 and Figure 10.} Transactions are not validated before being written to the backlog table ($\mathbf{S}$ in the whitepaper). There's some discussion of this in \href{https://github.com/bigchaindb/bigchaindb/issues/109}{Issue \#109 on GitHub}.
  \item \textbf{Section 4.5.} The data structures of transactions, blocks and votes have changed and will probably change some more. Their current schemas can be found in the \href{https://bigchaindb.readthedocs.io/en/latest/topic-guides/models.html}{BigchainDB Documentation}. \href{https://github.com/bigchaindb/bigchaindb/issues/342}{Issue \#342} lists other suggested changes.
  \item \textbf{Sections 4 and 6.} Blocks and votes are now stored in separate, append-only tables. (Votes used to be appended to lists inside blocks.) Votes are still used to form the links in the blockchain. See the discussion on \href{https://github.com/bigchaindb/bigchaindb/issues/368}{Issue \#368} and related issues on GitHub.
  \item \textbf{Section 5.2.} A BigchainDB node can have arbitrarily-large storage capacity (e.g. on a RAID array). Other factors limit the maximum storage capacity of a cluster (e.g. \href{https://bigchaindb.readthedocs.io/en/latest/nodes/node-requirements.html#memory-ram-requirements}{available RAM} and the \href{https://rethinkdb.com/limitations/}{maximum shard count of each table}).
  \item \textbf{Section 5.} We are planning to use a next-generation firewall, or something similar, around each node, so that each node can drop certain incoming requests, such as drop table, drop database, or update/delete block/vote.
\end{itemize}

\end{document}
