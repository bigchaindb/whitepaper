\section{Background: Traditional Blockchain Scalability}\label{sec:background}

This section discusses how traditional blockchains perform with respect to scalability, with an emphasis on Bitcoin.

\subsection{Technical Problem Description}
Technically speaking, a traditional blockchain is a distributed database (DB) that solves the “Strong Byzantine Generals” (SBG) problem \cite{koshy2014revolution}, the name given to a combination of the Byzantine Generals problem and the Sybil Attack problem.
In the Byzantine Generals problem \cite{lamport1982byzantine}, nodes need to agree on some value for a DB entry, under the constraint that the nodes may fail in arbitrary (malicious) ways.
The Sybil Attack problem \cite{douceur2002sybil} arises when one or more nodes figure out how to get unfairly disproportionate influence in the process of agreeing on a value for an entry.
It’s an “attack of the clones”—an army of seemingly independent voters actually working together to game the system.

\subsection{Bitcoin Scalability Issues}
Bitcoin has scalability issues in terms of throughput, latency, capacity, and network bandwidth.

\medskip
\noindent\textbf{Throughput.} The Bitcoin network processes just $1$ transaction per second (tps) on average, with a theoretical maximum of $7$ tps \cite{bitcoin2015scalability}.
It could handle higher throughput if each block was bigger, though right now making blocks bigger would lead to size issues (see Capacity and network bandwidth, below).
This throughput is unacceptably low when compared to the number of transactions processed by Visa ($2,000$ tps typical, $10,000$ tps peak) \cite{trillo2013visa}, Twitter ($5,000$ tps typical, $15,000$ tps peak), advertising networks ($500,000$ tps typical), trading networks, or email networks (global email volume is $183$ billion emails/day or $2,100,000$ tps \cite{sourabh2014email}).
An ideal global blockchain, or set of blockchains, would support all of these multiple high-throughput uses.

\medskip
\noindent\textbf{Latency.} Each block on the Bitcoin blockchain takes $10$ minutes to process.
For sufficient security, it is better to wait for about an hour, giving more nodes time to confirm the transaction.
By comparison, a transaction on the Visa network is approved in seconds at most.
Many financial applications need latency of $30$ to $100$~ms.

\medskip
\noindent\textbf{Capacity and network bandwidth.} The Bitcoin blockchain is about $70$~GB (at the time of writing); it grew by $24$~GB in $2015$ \cite{blockchaininfo2015blockchain_size}.
It already takes nearly a day to download the entire blockchain.
If throughput increased by a factor of $2,000$, to Visa levels, the additional transactions would result in database growth of $3.9$~GB/day or $1.42$~PB/year.
At $150,000$~tps, the blockchain would grow by $214$~PB/year (yes, petabytes).
If throughput were 1M tps, it would completely overwhelm the bandwidth of any node’s connection.

\subsection{Technology Choices Affecting Scalability}
The Bitcoin blockchain has taken some technology choices which hurt scaling:
\begin{enumerate}
 \item \textbf{Consensus Algorithm: POW.} Bitcoin’s mining reward actually incentivizes nodes to increase computational resource usage, without any additional improvements in throughput, latency, or capacity. A single confirmation from a node takes $10$ minutes on average, so six confirmations take about an hour. In Bitcoin this is by design; Litecoin and other altcoins reduce the latency, but compromise security.
 \item \textbf{Replication: Full.} That is, each node stores a copy of all the data; a “full node.” This copy is typically kept on a single hard drive (or in memory). Ironically, this causes centralization: as amount of data grows, only those with the resources to hold all the data will be able to participate.
\end{enumerate}

These characteristics prevent the Bitcoin blockchain from scaling up.

\subsection{Blockchain Scalability Efforts}
The Bitcoin / blockchain community has spent considerable effort on improving the performance of blockchains.
Appendix~\ref{appendix:blockchain_scalability} reviews various proposals in more detail.

Previous approaches shared something in common: they are all started with a block chain design then tried to increase its performance.
There’s another way: start with a ``big~data'' distributed database, then give it blockchain-like characteristics.